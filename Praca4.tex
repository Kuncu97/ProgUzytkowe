\documentclass{article}
\usepackage[utf8]{inputenc}
\usepackage{polski}
\usepackage{enumerate}
\usepackage[pdftex]{hyperref}
\usepackage{multirow}
\usepackage{makeidx}
\usepackage[tableposition=top]{caption}
\usepackage{graphicx}
\usepackage{amsmath}
\usepackage{amssymb}
\usepackage{amsfonts}
\usepackage{algorithmic}
\usepackage{algorithm2e}

\title{Algorytmy Graficzne}
\author{Szlegel - grupa V}
\date{2017-11-21}

\begin{document}

\maketitle
\begin{abstract}
Grafika komputerowa — dziedzina informatyki zajmująca się wykorzystaniem
technik komputerowych do celów wizualizacji artystycznej
oraz wizualizacji rzeczywistości. Grafika komputerowa jest obecnie
narzędziem powszechnie stosowanym w nauce, technice oraz rozrywce

\end{abstract}

\tableofcontents

\section{Nazwy kolorów w HTML}
Specyfikacja języka HTML 4.01 zawiera szesnaście nazwanych kolorów, zawartych
w tabeli 1.

\section{Algorytm Bresenhama}
Algorytm Bresenhama służy do rasteryzacji krzywych płaskich, czyli do jak
najlepszego ich obrazowania na siatce pikseli. Jack Bresenham w 1965 ro-\\

\begin{tabular}{l c r}
\bfseries Kolor & \bfseries{Nazwa po angielsku} & \bfseries HEX\\ \hline
cyjan & aqua & \#00FFFF \\
granatowy & navy & \#000080 \\
czarny & black & \#000000 \\
oliwkowy & olive & \#808000 \\
niebieski & blue & \#0000FF \\
fioletowy & violet & \#800080 \\ \hline 
\\ & Tabela 1: Kolory HTML
\end{tabular} \newline \\
ku w artykule [1] opracował metodę rasteryzacji odcinków, którą następnie
przystosowano do rysowania obiektów innego rodzaju (okręgów czy elips).

\subsection{Algorytm Bresenhama dla elipsy}
\subsubsection{Założenia}
\begin{enumerate}
    \item Elipsa ma osie zgodne z osiami układu współrzędnych,
    \item Półosie elipsy mają długości $a$ (wzdłuż osi $OX$) i $b$ (wzdłuż $OY$),
    \item Rozważamy elipsę w I ćwiartce układu współrzędnych
    \item Środkiem symetrii elipsy jest środek układu współrzędnych,
    \item Rysowanie elipsy zaczynamy od punktu (0,$b$),
    \item  W każdym kroku stawiamy symetrycznie 4 punkty elipsy,
        \begin{enumerate}[(a)]
            \item początkową osią wiodącą jest oś $OX$,
            \item w punkcie zmiany osi wiodącej, współczynnik nachylenia stycznej \\
                  do elipsy wynosi -1(tg135$^\circ$).
        \end{enumerate}
\end{enumerate}

\subsection{Algorytm i jego działanie}
Przybliżana elipsa ma równanie:
$$\frac{x^2}{a^2}+\frac{y^2}{b^2}-1=0.$$

O wyborze piksela decydować będzie wartość funkcji
$$F(x,y)=a^2b^2\left(\frac{x^2}{a^2}+\frac{y^2}{b^2}-1\right)=b^2x^2+a^2y^2-a^2b^2$$
w punkcie środkowym $M$ położonym pomiędzy alternatywnymi pikselami.\\
Gdy osią wiodąca jest $OX$ oblicza się
$$F(M)=F\left(x_i+1,y_i-\frac{1}{2}\right)$$

Przy wyborze następnego piksela (rysunek 1) $P_{i+1} = S$ czyli $x_{i+1} =$ \\
$x_i,y_{i+1}=y_i-1$ wartość zmiennej decyzyjnej wynosi: \\ \\
$d_{i+1}=F\left(x_{i+1}+\frac{1}{2},y_{i+1}-1\right)=$\\

$=b^2(x_{i+1}+1/2)^2+a^2(y_{i+1}-1)^2-a^2b^2=d_i-2a^2y_{i+1}+a^2$ \\
\rule[0.05cm]{\textwidth}{1pt} 
\textbf{Algorytm 1} Algorytm Bresenhama \\
\rule[0.05cm]{\textwidth}{1pt}
\textbf{Założenia:} Środek elipsy jest w (0,0), promienie $a,b \in\mathbb{N}$ \\
\textbf{Wynik:} Elipsa została wyświetlona

\begin{algorithmic}
\STATE{$i \leftarrow 0, j \leftarrow b, f \leftarrow 4b^2-4a^2 b+a^2$}
\STATE{writePixel$(i,j)$}
\WHILE{$b^2 i\leq a^2 j$}
\IF{$f>0$}
\STATE{$f \leftarrow f+8b^2 i-8a^2 j+12b^2+8a^2$}
\STATE{$j \leftarrow j-1$}
\ELSE{$f \leftarrow f+8b^2 i+12b^2$}
\ENDIF
\STATE{$i \leftarrow i+1$}
\STATE{writePixel$(i,j)$}
\ENDWHILE
\end{algorithmic}
\rule[0.05cm]{\textwidth}{1pt}

Powyższy krok został zaimplementowany w linijce 5 algorytmu 1.

\section{Kwaterniony}
\subsection{Kwaternion jako macierz rzeczywista}
Kwaterniony zdefiniowane są jako macierze z przestrzeni $M_{2\times2} (\mathbb{C}).$ Innym \\
sposobem zapisu macierzowego jest [2]:

$$ \begin{equation}
q= \left[\begin{array}{cccc}
         a & b & -d & -c\\
         -b & a & -c & d\\
         d & c & a & b\\
         c & -d & -b & a
         \end{array}\right] dla\ a,b,c,d \in \mathbb{R}
\end{equation} $$
\subsection{Wyznacznik i moduł}
Wyznacznik kwaternionu q, określonego przez (1):
$$det\ q=a^2+b^2+c^2+d^2$$


\end{document}