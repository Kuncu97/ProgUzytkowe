\documentclass[a4paper,12pt]{article}
\usepackage[MeX]{polski}
\usepackage[utf8]{inputenc}

%opening
\title{Formu�y matematyczne}
\author{Dariusz Szlegel}

\begin{document}

\maketitle

\begin{equation}
\lim_{n \to \infty} \sum_{k-1}^n\frac{1}{k^2}=\frac{\pi^2}{6} 
\label{eq:rownanie1}
\end{equation}

\begin{equation}
\prod_{i=2}^{n=i^2}=\frac{\lim_{n \to 4} \left (1+\frac{1}{n} \right)^n}{\sum k \left(\frac{1}{n}\right)}
\end{equation}
�atwo doprowadzi� jest r�wnanie 1 do r�wnania 2

\begin{equation}
\int_{2}^{\infty}\frac{1}{\log_{2}x}dx=\frac{1}{x}\sin x=1-\cos^2(x)
\label{eq:rownanie3}
\end{equation}

\begin{equation}
\left| \begin{array}{cccc}
a_{11} & a_{12} & \ldots & a_{1K}\\
a_{21} & a_{22} & \ldots & a_{2K}\\
\vdots & \vdots & \ddots & \vdots\\
a_{K1} & a_{K2} & \ldots & a_{KK}
\end{array} \right| * \left| \begin{array}{c}
x_{1}\\
x_{2}\\
\vdots\\
x_{K}
\end{array} \right| = \left| \begin{array}{c}
b_{1}\\
b_{2}\\
\vdots\\
b_{K}
\end{array} \right|
\label{eq:rownanie4}
\end{equation}

\begin{equation}
\left(a_1=a_1(x)\right)\wedge\left(a_2=a_2(x)\right)\wedge\ldots\wedge\left(a_k=a_k(x)\right)\Rightarrow\left(d=d(u)\right)
\label{eq:rownanie5}
\end{equation}

\begin{equation}
\left[x\right]_A=\{y\in U : a(x) = a(y), \forall a \in A\}, \ where \ the \ central \ object \ x \in U
\label{eq:rownanie6}
\end{equation}

\end{document}